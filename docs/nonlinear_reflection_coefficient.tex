\documentclass[letterpaper]{article}
\usepackage{amsmath}
\usepackage{lmodern}

\begin{document}
\section{Preamble}
Our interest is to calculate the nonlinear reflection coefficient, 
$\mathcal{R}$. We start with equation (2) of \cite{mejiaPRB02}

\begin{equation}
\mathcal{R}_{iF} = 
\frac{32\pi^{3}\omega^{2}}{(n_{o}e)^{2}c^{3}\cos^{2}\theta}
\left\vert T^{v\ell }_{F}T^{\ell b}_{F}(t^{v\ell }_{i}t^{\ell b}_{i})^{2}r_{iF}\right\vert^{2},
\end{equation}

where $i=s$ or $p$ is the incoming polarization at $\omega$, $F=S$ or $P$ the 
outgoing polarization at $2\omega$. $\theta$ is the angle of incidence, 
$n_{o}$ is the electron density, $e$ is the electron charge, and $c$ is the 
speed of light. $T$ and $t$ are the Fresnel factors that give the transmitted 
fields at the vacuum--surface $(vs)$ or the surface--bulk $(sb)$ interfaces. 

Thus, we have four different cases for $\mathcal{R}_{iF}$: $\mathcal{R}_{pP}$, 
$\mathcal{R}_{sP}$, $\mathcal{R}_{pS}$, and $\mathcal{R}_{sS}$. Let us derive 
the most explicit equations for these cases.

\section{Unit Analysis}

The units for $\mathcal{R}_{iF}$ should be reported in 
$\text{cm}^{2}/\text{W}$\cite{mejiaPRB02}. All terms inside the absolute value 
bars are adimensional, implying that 
$n_{o}e = \frac{\text{pm}^{2}}{\text{V}}$ Outside, we have

\begin{equation*}
\frac{32\pi^{3}\omega^{2}}{(n_{o}e)^{2}c^{3}\cos^{2}\theta} = 
\frac{[\frac{1}{\text{s}}]^{2}}{[\frac{\text{C}}{\text{m}^{3}}]^{2}[\frac{\text{m}}{\text{s}}]^{3}} = 
\frac{\frac{1}{\text{s}^{2}}}{\frac{\text{C}^{2}}{\text{m}^{6}}\frac{\text{m}^{3}}{\text{s}^{3}}} = 
\frac{\frac{1}{\text{s}^{2}}}{\frac{\text{C}^{2}}{\text{m}^{3}\text{s}^{3}}} = 
\frac{\text{m}^{3}\text{s}^{3}}{\text{C}^{2}\text{s}^{2}} = 
\frac{\text{m}^{3}\text{s}}{\text{C}^{2}} 
\end{equation*}

\section{Fresnel Factors}

The Fresnel factors are given by
\begin{align}
&t^{v\ell}_{s}(\omega) = \frac{2\cos\theta}{\cos\theta + k_{z\ell}(\omega)},\label{tvss}\\
&t^{v\ell}_{p}(\omega) = \frac{2\cos\theta}{\epsilon_{\ell}(\omega)\cos\theta + k_{z\ell}(\omega)},\label{tvsp}\\
&t^{\ell b}_{s}(\omega) = \frac{2k_{z\ell}(\omega)}{k_{z\ell}(\omega) + k_{zb}(\omega)},\label{tsbs}\\
&t^{\ell b}_{p}(\omega) = \frac{2k_{z\ell}(\omega)}{\epsilon_{b}(\omega)k_{z\ell}(\omega) + \epsilon_{\ell}(\omega)k_{zb}(\omega)}.\label{tsbp}
\end{align}
where $k_{zj}(\omega) = \sqrt{\epsilon_{j}(\omega) - \sin^{2}\theta}$ for $j=\ell$ (surface) or $b$ (bulk). $\epsilon_{b}\,(\epsilon_{\ell})$ is the bulk (surface) dielectric function. The Fresnel factors for the outgoing fields are $T=t(2\omega)$. We can derive these starting with $k_{zj}(\omega)$ and substitute into the Fresnel factors. From \eqref{tvss},

\begin{align*}
t^{vs}_{s}(\omega) = \frac{2\cos\theta}{\cos\theta + k_{zs}(\omega)} &= \frac{2\cos\theta}{\cos\theta + (w/c)(\epsilon_{s}(\omega) - \sin^{2}\theta)^{1/2}} \nonumber \\
&=\frac{2\cos\theta}{\cos\theta + (w/\sqrt{2}c)(2\epsilon_{s}(\omega) + \cos 2\theta + 1)^{1/2}}.
\end{align*}

From \eqref{tvsp},

\begin{align*}
t^{vs}_{p}(\omega) &= \frac{2\cos\theta}{\epsilon_{s}\cos\theta + k_{zs}(\omega)} \nonumber \\
&=\frac{2\cos\theta}{\epsilon_{s}\cos\theta + (w/\sqrt{2}c)(2\epsilon_{s}(\omega) + \cos 2\theta + 1)^{1/2}}.
\end{align*}

From \eqref{tsbs},

\begin{align*}
t^{sb}_{s}(\omega) &= \frac{2k_{zs}(\omega)}{k_{zs}(\omega) + k_{zb}(\omega)} \nonumber \\
&= \frac{2(\epsilon_{s}(\omega) - \sin^{2}\theta)^{1/2}}{(\epsilon_{s}(\omega) - \sin^{2}\theta)^{1/2} + (\epsilon_{b}(\omega) - \sin^{2}\theta)^{1/2}} \nonumber \\
&=\frac{2(2\epsilon_{s}(\omega) + \cos 2\theta + 1)^{1/2}}{(2\epsilon_{s}(\omega) + \cos 2\theta + 1)^{1/2} + (2\epsilon_{b}(\omega) + \cos 2\theta + 1)^{1/2}}.
\end{align*}

Lastly, from \eqref{tsbp},

\begin{align*}
t^{sb}_{s}(\omega) &= \frac{2k_{zs}(\omega)}{\epsilon_{b}(\omega)k_{zs}(\omega) + \epsilon_{s}(\omega)k_{zb}(\omega)} \nonumber \\
&=\frac{2(2\epsilon_{s}(\omega) + \cos 2\theta + 1)^{1/2}}{\epsilon_{b}(\omega)(2\epsilon_{s}(\omega) + \cos 2\theta + 1)^{1/2} + \epsilon_{s}(\omega)(2\epsilon_{b}(\omega) + \cos 2\theta + 1)^{1/2}}.
\end{align*}

\section{$r_{iF}$ Terms}

Finally, the $r_{iF}$ terms are

\begin{align*}
r_{pP} = \sin\theta\epsilon_{b}(2\omega)[\sin^{2}\theta\epsilon^{2}_{b}(\omega)\chi_{\perp\perp\perp} &+ k^{2}_{zb}(\omega)\epsilon^{2}_{s}(\omega)\chi_{\perp\parallel\parallel}] \nonumber \\
+ \epsilon_{s}(\omega)\epsilon_{s}(2\omega)k_{zb}(\omega)k_{zb}(2\omega)&[-2\sin\theta\epsilon_{b}(\omega)\chi_{\parallel\parallel\perp}&\\
&+ k_{zb}(\omega)\epsilon_{s}(\omega)\chi_{\parallel\parallel\parallel}\cos(3\phi)], \nonumber
\end{align*}

\begin{equation*}
r_{sP} = \sin\theta\epsilon_{b}(2\omega)\chi_{\perp\parallel\parallel} - k_{zb}(2\omega)\epsilon_{s}(2\omega)\chi_{\parallel\parallel\parallel}\cos(3\phi),
\end{equation*}

\begin{equation*}
r_{pS} = -k^{2}_{zb}(\omega)\epsilon^{2}_{s}(\omega)\chi_{\parallel\parallel\parallel}\sin(3\phi),
\end{equation*}

\begin{equation*}
r_{sS} = \chi_{\parallel\parallel\parallel}\sin(3\phi),
\end{equation*}

where $\phi$ is the azimuthal angle. $\chi$ is the second-order susceptibility tensor with the following relations\footnote{for this symmetry group.},

\begin{align}
\chi_{\perp\perp\perp}&\equiv\chi_{zzz}, \nonumber \\
\chi_{\perp\parallel\parallel}&\equiv\chi_{zxx}=\chi_{zyy}, \nonumber \\
\chi_{\parallel\parallel\perp}&\equiv\chi_{xxz}=\chi_{yyz}, \nonumber \\
\chi_{\parallel\parallel\parallel}&\equiv\chi_{xxx}=-\chi_{xyy}=-\chi_{yyx}.
\end{align}

Expanding these expressions, we obtain

\begin{align*}
r_{pP} = \sin\theta\epsilon_{b}(2\omega)[\sin^{2}\theta\epsilon^{2}_{b}(\omega)\chi_{\perp\perp\perp} &+ k^{2}_{zb}(\omega)\epsilon^{2}_{s}(\omega)\chi_{\perp\parallel\parallel}] \nonumber \\
+ \epsilon_{s}(\omega)\epsilon_{s}(2\omega)k_{zb}(\omega)k_{zb}(2\omega)&[-2\sin\theta\epsilon_{b}(\omega)\chi_{\parallel\parallel\perp}&\\
&+ k_{zb}(\omega)\epsilon_{s}(\omega)\chi_{\parallel\parallel\parallel}\cos(3\phi)], \nonumber
\end{align*}

\begin{equation*}
r_{sP} = \sin\theta\epsilon_{b}(2\omega)\chi_{\perp\parallel\parallel} - k_{zb}(2\omega)\epsilon_{s}(2\omega)\chi_{\parallel\parallel\parallel}\cos(3\phi),
\end{equation*}

\begin{equation*}
r_{pS} = -k^{2}_{zb}(\omega)\epsilon^{2}_{s}(\omega)\chi_{\parallel\parallel\parallel}\sin(3\phi),
\end{equation*}

\begin{equation*}
r_{sS} = \chi_{\parallel\parallel\parallel}\sin(3\phi),
\end{equation*}

\bibliographystyle{plain}
\bibliography{nrc}
\end{document}
